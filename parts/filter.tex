% vim:ts=2 sw=2 noet spell tw=78:

\section{Filters}

\subsection{Inverse Signal}

$f[k]$ is said to be the \emph{inverse} of $g[k]$ if $F(z)G(z) = 1$.
If $f$ is an impulse response of a system, then $g$ is the \emph{inverse
filter} of said system. $f$ may not be unique, so if there are many ($f_i$),
any linear combination of inverse filters ($\alpha f_1 + \beta f_2$) is still
an inverse filter. Therefore $1/F(z)$ does not represent a unique signal! One
common way to solve this ambiguity is to enforce that $f$ and $g$ are right
sided, then there is only one (right sided) inverse. The same also applies to
left sided signals.

% TODO: computations, stable vs unstable inverse
% To compute the inverse one can use long division. The left sided signal is
% obtained by performing the long division of 

\subsection{Linear Equalization}

A linear equalizer for $H(z)$ is a $G(z)$ such that
\[
	H(z) G(z) = z^{-L} + E(z) \approx z^{-L}
\]
for a delay $L \geq 1$, where $E(z)$ is an error term that can be made
arbitrarily small by choosing an appropriate $L$.

\subsection{Decision-Feedback Equalization}

A DFE for $H(z)$ uses the fact that
\[
	H(z) = H_1(z) + z^{-L-1} H_2(z), \text{ where }
\]
\[
	H_1(z) = \sum_{k=0}^L h[k] z^{-k}, \quad
	H_2(z) = z^{L+1}\sum_{k=L+1}^\infty h[k] z^{-k}.
\]
The equalizer has a forward and backwards filter, $G_f(z)$ and $G_b(z)$
respectively, that are given by
\[
	G_b(z) = -H_2(z), \quad
	G_f(z) = z^L F(z) \mod z.
\]
where $F(z)$ is the stable (left sided) inverse of $H_1(z)$ truncated at $L$
terms.
